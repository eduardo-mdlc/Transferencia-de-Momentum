\chapter{Capitulo 1}
\justifying
\textbf{Introducción}: Consta de las siguientes secciones:
 	\begin{enumerate}
 		 \item Definiciones. 
 		 \item Ecuaciones de Conservación.
 		  \item Ecuacione Constitutiva. Ley de Fick.
 		   \item Balance de Masa. 
 		   \item Calculo de la Difusividad. 
 	   \end{enumerate} 
 	   \section{Definiciones} 
 	   \begin{itemize}
 	  \item \emph{$\rho_i$}: Concentración másica de la especie i (masa de i/volumen)
 	  \item $C_i$=$\frac{\rho_i}{\mu_i}$: Concentración molar de la especie i (moles/volumen)
 	  \item $w_i=\frac{\rho_i}{\rho}$: fracción masica de i ($\rho$ es la densidad de la solución)
 	  \item $x_i=\frac{C_i}{C}$: fracción molar ($C$ es la densidad molar total de la solución)
 	  \item \emph{$ \underline{v}_i$}: velocidad de la especie i con respecto a un sistema de coordenadas fijo. La velocidad promedio local \emph{$ \underline{v}$} se define como:
 	 \end{itemize}
 	 \begin{equation}
 	 \emph{$ \underline{v}=\frac{\sum_{i=1}^n \rho_i  \underline{v}_i}{\sum_{j=1}^n \rho_i}$}
 	 \tag{1.1}
     \label{eq_1.1}
 	  \end{equation}
 	  \emph{$\rho  \underline{v}$} = flux de masa. La velocidad promedio molar es:
 	   \begin{equation}
 	  	\emph{$ \underline{v}^*=\frac{\sum_{i=1}^n C_i  \underline{v_i}}{\sum_{i=1}^n C_i}$}
 	  	\tag{1.2}
        \label{eq_1.2}
 	  \end{equation}
 	
$C\underline{\emph{v}}=$flux molar de masa.
 \begin{equation}
 \underline{	{v}_i} -\underline{{v} }=\quad \text{velocidad de difusión de $i$ con respecto a ${v}$}. \tag{1.3}
 \label{eq_1.3}
 \end{equation}
 
 \begin{equation}
 	\underline{{v}_i} -\underline{{v}^*}=\quad \text{velocidad de difusión de $i$ con respecto a $\vec{v}^*$}. \tag{1.4} \label{eq_1.4}
 \end{equation}
 
 Con respecto a ejes estacionarios, los flujos de masa se definen de la siguiente manera:
 \begin{equation}
 	\underline{n}_i = \rho_i \underline{v}_i \quad \text{másico} \tag{1.5}
    \label{eq_1.5}
 \end{equation}
 \begin{equation}
 	\underline{N}_i = C_i \underline{v}_i \quad \text{molar} \tag{1.6}\label{eq_1.6}
 \end{equation}
 
 Con respecto a las velocidades promedio:
 \begin{equation}
 		\underline{{j}_i} = \rho_i (	\underline{{v}_i} -\underline{v}) \tag{1.7}\label{eq_1.7}
 \end{equation}
 \begin{equation}
 	\underline{{J}_i} = C_i (\underline{{v}_i} -	\underline{v}) \tag{1.8} \label{eq_1.8}
 \end{equation}
 
 Con respecto a las velocidades molares promedio:
 \begin{equation}
 	\underline{j}_i^* = \rho_i (\underline{v}_i - \underline{v}^*) \quad \text{másico} \tag{1.9} \label{eq_1.9}
 \end{equation}
 \begin{equation}
 	\underline{J}_i^* = C_i (\underline{v}_i - \underline{v}^*) \quad \text{molar} \tag{1.10} \label{eq_1.10}
 \end{equation}
 
 Relaciones entre los flujos molares. A partir de las ecs. \textbf{\eqref{eq_1.2}} y \textbf{\eqref{eq_1.10}} :
 \begin{equation}
 	\underline{J}_i^* = C_i (\underline{v}_i - \underline{v}^*) = C_i \underline{v}_i^* - \frac{C_i}{C} \sum_{j=1}^n C_j \underline{v}_j \tag{1.11} \label{eq_1.11}
 \end{equation}
 
 Por las relaciones (ecs. \textbf{\eqref{eq_1.6}} y $x_i$):
 \begin{equation}
 	\underline{J}_i^* = \underline{v}_i - x_i \sum_{j=1}^n x_j \underline{v}_j \tag{1.12}\label{eq_1.12}
 \end{equation}
 
 Para un sistema binario=$\underline{J}_A^*=\underline{N}_A-x_A (\underline{N_A}+\underline{N_B})$
 
 Para los fluxes másicos con respecto a \underline{\emph{v}} tenemos que: $\underline{J}_A=\underline{n}_A-w_A (\underline{n_A}+\underline{n_B})$

 A partir de la ec. \textbf{\eqref{eq_1.12}} se obtiene que:
 \begin{equation}
 	\sum_{i=1}^n \underline{J}_i^* =0 \tag{1.13}\label{eq_1.13}
 \end{equation}
 
 y para una mezcla binaria se obtiene:
 \begin{equation}
 	\underline{J}_A^* + \underline{J}_B^* = 0 \tag{1.14}\label{eq_1.14}
 \end{equation}
 
 Igualmente, la ec. \textbf{\eqref{eq_1.1}} puede ser escrita como:	
 
$\rho \underline{\emph{v}}=\rho_A \underline{\emph{v}}_A+\rho_B \underline{\emph{v}}_B=\underline{n}_B+\underline{n}_B$
 
 y la ec. \textbf{\eqref{eq_1.7}} puede ser expresada como:
 
 $ \underline{j}_A=	\rho_A \underline{v}_A + \rho_A \underline{v} \Rightarrow \underline{n}_A = \underline{j}_A + \rho_A \underline{v} $
 
 \section{Ecuaciones de Conservación}
 \textbf{Mezclas Binarias}
 
 La ecuación de continuidad para el componente $A$ en una mezcla binaria está dada por la siguiente ecuación:
 \begin{equation}
 	\frac{\partial \rho_A}{\partial t} + \nabla \cdot \underline{n}_A = r_A \tag{1.15}\label{eq_1.15}
 \end{equation}
 
 donde $\underline{n}_A=\rho_A \underline{v}_A$ es el flujo de masa.		 Similarmente, para el componente $B$:
 \begin{equation}
 	\frac{\partial \rho_B}{\partial t} + \nabla \cdot \underline{n}_B = r_B \tag{1.16}\label{eq_1.16}
 \end{equation}
 
 La adición de \textbf{\eqref{eq_1.15}} y \textbf{\eqref{eq_1.16}} (en donde  $\underline{n}_A + \underline{n}_B = \rho \underline{v}$ y $r_A + r_B = 0$)
 \begin{equation}
 	\frac{\partial \rho}{\partial t} + \nabla \cdot \rho \underline{v} = 0 \tag{1.17} \label{eq_1.17}
 \end{equation}
 
 que es la ecuación de continuidad o conservación de masa de la mezcla. En caso en que $\rho$ es constante:
 \begin{equation}
 	\nabla \cdot \underline{v} = 0 \tag{1.18}\label{eq_1.18}
 \end{equation}
 
 En términos de las variables molares:
 \begin{equation}
 	\frac{\partial C_A}{\partial t} + (\nabla \cdot \underline{N_A}) = R_A \tag{1.19} \label{eq_1.19}
 \end{equation}
 \begin{equation}
 	\frac{\partial C_B}{\partial t} + \nabla \cdot \underline{N}_B = R_B \tag{1.20} \label{eq_1.20}
 \end{equation}
 
 Adicionando:
 \begin{equation}
 	\frac{\partial C}{\partial t} + \nabla \cdot C {v}^* = R_A + R_B \tag{1.21} \label{eq_1.21}
 \end{equation}
 
 La ecuación \textbf{\eqref{eq_1.21}} supone que $\underline{N}_A +\underline{ N}_B = C \underline{v}^*$ y $R_A + R_B \neq 0$ si los moles no se conservan. Para un fluido de $C$ constante:
 \begin{equation}
 	\nabla \cdot \underline{v}^* = \frac{1}{C} (R_A + R_B) \tag{1.22}
    \label{eq_1.22}
 \end{equation}
 
 En el \underline{Apéndice A}, la ecuación \textbf{\eqref{eq_1.19}} se expresa en los tres sistemas de coordenadas.



 \section{Ecuación constitutiva: Ley de Fick}

Definiendo la difusividad másica $\mathscr{D}_{AB}=\mathscr{D}_{BA}$ en un sistema binario, la expresión del flux de masa es:
\begin{equation}
	\underline{J}_A^* = - \emph{C} \mathscr{D}_{AB} \nabla x_A  \tag{1.23}
    \label{eq_1.23}
\end{equation}
Análogamente,
\begin{equation}
	\underline{j}_A = - \rho \mathscr{D}_{AB} \nabla w_A  \tag{1.24}\label{eq_1.24}
\end{equation}

Esta es la primera ley de Fick de difusión. En términos de $N_A$ relativo a coordenadas estacionarias,
\begin{equation}
	\underline{N}_A = x_A (\underline{N}_A + \underline{N}_B) - \emph{C} \mathscr{D}_{AB} \nabla x_A  \tag{1.25}\label{eq_1.25}
\end{equation}


\begin{equation}
	\underline{n}_A = w_A (\underline{n}_A + \underline{n}_B) - \rho \mathscr{D}_{AB} \nabla w_A  \tag{1.26}\label{eq_1.26}
\end{equation}

 \section{Balance de Masa}
 Sustituyendo la ecuación \textbf{\eqref{eq_1.26}} en la ecuación \textbf{\eqref{eq_1.15}} se obtiene:
 \begin{equation}
 	\frac{\partial \rho_A}{\partial t} + \nabla \cdot \rho_A \underline{v} = \nabla \cdot \rho \mathscr{D}_{AB} \nabla w_A + r_A \tag{1.27}\label{eq_1.27}
 \end{equation}
 
 Similarmente, sustituyendo la ecuación \textbf{\eqref{eq_1.25}} en la ecuación \textbf{\eqref{eq_1.19}} se obtiene:
 \begin{equation}
 	\frac{\partial {C}_A}{\partial t} + \nabla \cdot C_A \underline{v}^* = \nabla \cdot C \mathscr{D}_{AB} \nabla x_A + R_A  \tag{1.28}\label{eq_1.28}
 \end{equation}
 
 Si $\rho$ y $\mathscr{D}_{AB}$ son constantes, entonces:
 \begin{equation}
 	\frac{\partial \rho_A}{\partial t} + \rho  (\nabla\cdot \underline{v})+\underline{v} \cdot \nabla \rho_A = \mathscr{D}_{AB} \nabla^2 \rho_A + r_A  \tag{1.29}\label{eq_1.29}
 \end{equation}
 
 donde $\nabla \cdot v = 0$ (Ec. \textbf{\eqref{eq_1.18}}). Dividiendo \textbf{\eqref{eq_1.29}} entre $M$ se obtiene:
 \begin{equation}
 	\frac{\partial C_A}{\partial t} + \underline{v} \cdot \nabla C_A = \mathscr{D}_{AB} \nabla^2 C_A + R_A  \tag{1.30} \label{eq_1.30}
 \end{equation}
 
 Esta ecuación se emplea comúnmente en soluciones diluidas líquidas. Si $C$ y $\mathscr{D}_{AB}$ son constantes, la ecuación \textbf{\eqref{eq_1.28}} se convierte en:
 \begin{equation}
 	\frac{\partial C_A}{\partial t} +C_A (\nabla \cdot \underline{v}^*)+ \underline{v}^*\cdot\nabla C_A  = \mathscr{D}_{AB} \nabla^2 C_A + R_A \tag {1.31}\label{eq_1.31}
 \end{equation}
 
 Como: $	\nabla \cdot \underline{v^*}= \frac{1}{c} (R_A + R_B) \quad \text{(Ver ec. \textbf{\eqref{eq_1.21})}}$
 
 Entonces:
 \begin{equation}
 	\frac{\partial C_A}{\partial t} + \underline{v}^* \cdot\nabla C_A=\mathscr{D}_{AB} \nabla^2 C_A+R_A-\frac{C_A}{C}(R_A+R_B) \tag{1.32}\label{eq_1.32}
 \end{equation}
 Si la velocidad es cero y no hay reacción química, se obtiene:
  \begin{equation}
 	\frac{\partial C_A}{\partial t}=\mathscr{D}_{AB} \nabla^2 C_A \tag{1.33}\label{eq_1.33}
 	\end{equation}
 	
 \section{Cálculo de la difusividad}
 \subsection{Teoría de la difusión de gases a baja densidad.}
 		Considerar un gas con moléculas A y A$^*$	que tienen la misma masa $m_A$
 		y el mismo tamaño y forma. Se plantea determinar la difusividad másica $\mathscr{D}_{AA^*}$
 		de un sistema de moléculas rígidas de diámetro $d_A$. Utilizando las definiciones obtenidas en el libro y en la teoría cinética de gases:
 	
 	Velocidad promedio:
 		\begin{equation}
 			\langle v \rangle = \sqrt{\frac{8k_B	T}{\pi m}}  \tag{1.34}\label{eq_1.34}
 		\end{equation}
 		$k_B=$Constante de Boltzman
 	
 	Frecuencia de colisiones:
 		\begin{equation}
 			Z = \frac{1}{4} n \langle v \rangle  \tag{1.35}\label{eq_1.35}
 		\end{equation}
 		Trayectoria libre media
 		\begin{equation}
 			\lambda = \frac{1}{\sqrt{2} \pi d^2 n}  \tag{1.36}\label{eq_1.36}
 		\end{equation}
 		Dado un dominio que contiene planos con encuadraciones diferentes separados por una distancia a, la relación entre a y $\lambda$ es:
 		 		\begin{equation}
 		 			a=\frac{2}{3} \lambda \tag{1.37}\label{eq_1.37}
 		 			\end{equation}
        \begin{figure}[h]

        \centering\includegraphics[width=0.6\linewidth]{Capitulo1/Imagenes/Fig_1.1.jpeg}
        \caption{Transferencia molecular de A del plano (y-a) al plano(y)}
        \label{fig:Fig_1.1}

\end{figure}

        
        El flux de masa de A efectivo entre el plano (y+a) y el plano (y-a) es:
        \begin{equation}
N_{A_{y}} = \frac{1}{\vec{N}} \left[ n x_A  v_y^* \bigg|_{y} + \frac{1}{4}n x_A \langle v \rangle \bigg|_{y-a}- \frac{1}{4} nx_A\langle v \rangle \bigg|_{y+a} \right]
\tag{1.38} \label{eq_1.38}
\end{equation}

Si el perfil de concentraciones es lineal, entonces:

\begin{equation}
x_A \bigg|_{y-a} = x_A \bigg|_{y-a} - \frac{2}{3}\lambda \frac{dx_A}{dy}
\tag{1.39} \label{eq_1.39}
\end{equation}

\begin{equation}
x_A \bigg|_{y+a} = x_A \bigg|_{y+a} + \frac{2}{3}\lambda \frac{dx_A}{dy}
\tag{1.40} \label{eq_1.40}
\end{equation}

Ya que \( Cv^* = N_A + N_B \), las ecs. \textbf{\eqref{eq_1.37}}, \textbf{\eqref{eq_1.38}}, \textbf{\eqref{eq_1.39}} y \textbf{\eqref{eq_1.40}} se combinan para dar:

\begin{equation}
N_{A_{y}} = x_A \left( N_{A_y} + N_{A_y}^* \right) - \frac{2}{3}C \langle v \rangle \lambda \frac{dx_A}{dy} \tag{1.41}\label{eq_1.41}
\end{equation}

De acuerdo con la ley de Fick:

\begin{equation}
\mathscr{D}_{AB} = \frac{1}{3} \langle v \rangle \lambda  \tag{1.42}\label{eq_1.42}
\end{equation}

Utilizando la ley de los gases ideales ( $P = CRT = nk_BT $) y las expresiones para \( \langle v \rangle \) y \( \lambda \), se obtiene para la autodifusión:

\begin{equation}
\mathscr{D}_{AA}^* = \frac{2}{3} \left( \frac{k_B^3}{\pi^3 m_A} \right)^{1/2} \frac{T^{3/2}}{P d_A^2}\tag{1.43}\label{eq_1.43}
\end{equation}

La ecuación \textbf{\eqref{eq_1.42}} representa la difusividad másica de una mezcla de dos especies de esferas rígidas de igual masa y diámetro. En el caso de diferentes masas y diámetros, la ecuación \textbf{\eqref{eq_1.43}} se generaliza:

\begin{equation}
\mathscr{D}_{AB} = \frac{2}{3} \left( \frac{k_B^3}{\pi^3} \right) \left( \frac{1}{2 m_A} + \frac{1}{2 m_B} \right)^{1/2} \frac{T^{3/2}}{P \left( \frac{d_A + d_B}{2} \right)^2}\tag{1.44}\label{eq_1.44}
\end{equation}

La teoría de Chapman-Enskog proporciona expresiones más exactas del coeficiente de difusión:

\begin{equation}
c \mathscr{D}_{AB} = (2.625 \times 10^{-5}) \frac{\sqrt{T(\frac{1}{M_A}+\frac{1}{M_B})}}{\sigma_{AB}^2 \Omega_{\mathscr{D}_{AB}}}\tag{1.45}\label{eq_1.45}
\end{equation}
Si C está dada por la ley de los gases ideales, entonces:

\begin{equation}
\mathscr{D}_{AB} = 0.00186 \sqrt{T^3(\frac{1}{M_A}+\frac{1}{M_B})}\frac{1}{P \sigma_{AB}^2 \Omega_{\mathscr{D}_{AB}}}  \tag{1.46}\label{eq_1.46}
\end{equation}

\(\Omega_{\mathscr{D}_{AB}}\) es una función adimensional de la temperatura y del potencial intermolecular entre moléculas \( A \) y \( B \). Este potencial se aproxima por el de Lennard-Jones:

\begin{equation}
\varphi_{AB}=4 \epsilon_{AB}[(\frac{\sigma_{AB}}{r})^{12}-(\frac{\sigma_{AB}}{r})^6]  \tag{1.47}\label{eq_1.47}
\end{equation}

En el Apéndice C se pueden encontrar tablas de \( \Omega_{\mathscr{D}_{AB}} \) en función de \( k_B / \epsilon_{AB} \). Para esferas rígidas, \( \Omega_{\mathscr{D}_{AB}} = 1 \). En mezclas binarias:

\begin{equation}
\sigma_{AB} = \frac{1}{2} (\sigma_A+\sigma_B)\tag{1.48}\label{eq_1.48}
\end{equation}

\begin{equation}
\epsilon_{AB} = \sqrt{\epsilon_A \epsilon_B}\tag{1.49}\label{eq_1.49}
\end{equation}

Las constantes $\sigma_{AB}$ y \( \epsilon_{AB} \) pueden ser identificadas en el Apéndice. En el caso de pares isotópicos:$\sigma_{AA^*}=\sigma_A=\sigma_A^*$  y  $\epsilon_{AA^*}=\epsilon_A=\epsilon_A^*$.




Si \( M_A = M_B \), la ecuación \eqref{eq_1.45} se reduce a:

\begin{equation}
C\mathscr{D}_{AA^*} = 3.2 \times 10^{-5}\sqrt{\frac{T}{M_A}} \frac{1}{\sigma_A \Omega_{\mathscr{D}_{AA^*}}} \tag{1.50}\label{eq_1.50}
\end{equation}

Definiendo el número de Schmidt:

\begin{equation}
Sc = \frac{\mu / \rho}{\mathscr{D}_{AB}} = \frac{\nu}{\mathscr{D_{AB}}} (\text{0.2 - 5 en gases}) \tag{1.51}\label{eq_1.51}
\end{equation}

Tenemos que:

\begin{equation}
\frac{\nu}{\mathscr{D}_{AA^*}} = \frac{5}{6}\frac{\Omega_{\mathscr{D}_{AA^*}}}{ \Omega_\mu}\tag{1.52}\label{eq_1.52}
\end{equation}

Por lo que: $\mathscr{D}_{AA^*} \approx 1.32 \nu$
\subsection{Dependencia de la difusividad de la presión y temperatura (PyT)}
En sistemas de mezclas de gases a bajas presiones,  $\mathscr{D}_{AB}$ es inversamente proporcional a P, directamente proporcional a T e independiente de la concentración de un par de gases.
Utilizando estados correspondientes y la teoría cinética, \( \mathscr{D}_{AB} \) está dado por la siguiente ecuación:
\setcounter{figure}{50}
\begin{equation}
\frac{P \mathscr{D}_{AB}}{(P_{C_A} P_{C_{B}})^{1/5} (T_{{C_B}} T_{C_{B}})^{5/12} \left( \frac{1}{M_A} + \frac{1}{M_B} \right)^{1/2}} = a \left( \frac{T}{T_{C_A} T_{C_B}} \right)^b\tag{1.53}\label{eq_1.53}
\end{equation}

Gases no polares: \( a = 2.745 \times 10^{-4} \) y \( b = 1.823 \)  

H$_2$O + gas no polar: \( a = 3.64 \times 10^{-4} \) y \( b = 2.334 \)

En la figura \textbf{\eqref{fig:Fig_1.51}} se grafica \( C \mathscr{D}_{AA^*}\) como función de \( T_R \) y \( P_R \) (T y P reducidas), \( (C \mathscr{D}_{AA^*})_R=\frac{C\mathscr{D}_{AA^*}}{(C\mathscr{D}_{AA^*)_c}} \) .
$((\mathscr{D})_c \text{es la cantidad crítica})$

$(C\mathscr{D}_{AA^*})_c$ coeficiente de difusión crítico puede ser estimado por la siguiente ecuación:

\begin{equation}
(C\mathscr{D}_{AA^*})_c = 2.96 \times 10^{-6} \left( \frac{1}{M_A} + \frac{1}{M_{A^*}} \right)^{1/2} \frac{P_{cA}^{1/3}}{T_{cA}^{1/6}}\tag{1.54}\label{eq_1.54}
\end{equation}

En el caso de gases \( A \) y \( B \), la ecuación \textbf{\eqref{eq_1.54}} puede reducirse a la aproximación siguiente:

\begin{equation}
(C \mathscr{D}_{AA^*}) = 2.96 \times 10^{-6} \left( \frac{1}{M_A} + \frac{1}{M_B} \right)^{1/2}  \frac{(P_{cA} P_{cB})^{1/3}}{(T_{cA} T_{cB})^{1/12}} \tag{1.55}\label{eq_1.55}
\end{equation}

Correspondientemente, en el caso de A y B en la figura \textbf{\eqref{fig:Fig_1.51}}
$T_R=\frac{T}{\sqrt{T_{cA}T_{cB}}}$ y $P_R=\frac{P}{\sqrt{P_{cA}P_{cB}}}$
\begin{figure}[H]
\centering
        \includegraphics[width=0.7
        \linewidth]{Capitulo1/Imagenes/Fig_1.51.jpeg}
        \caption{Mezclador estático (a) y mezclador dinámico (b)}
        \label{fig:Fig_1.51}

\end{figure}

\subsubsection{Teoría de la difusión en líquidos binarios}
La teoría hidrodinámica se basa en la ecuación de Nernst-Einstein, que postula que la difusividad de una partícula A (el soluto) en un medio estacionario B está expresado como:

\begin{equation}
\mathscr{D}_{AB} = k_B T \frac{U_A}{F_A}\tag{1.56}\label{eq_1.56}
\end{equation}

en donde $U_A/F_A $ es la movilidad de la partícula A. Si A es esférica y además existe deslizamiento en la interfase sólido-fluido, se obtiene:

\begin{equation}
\frac{U_A}{F_A} = \frac{(3 \mu_B + R_A \beta_{AB})}{({2 \mu_B + R_A \beta_{AB}} )}  \frac{1}{6 \pi \mu_B R_A}\tag{1.57}\label{eq_1.57}
\end{equation}

en donde \( \mu_B \) es la viscosidad del solvente, \( R_A \) es el radio de la partícula y \( \beta_{AB} \) es el coeficiente de fricción interfacial. Hay 2 casos:

a)-.$( \beta_{AB} = \infty )$ (sin deslizamiento).
   En este caso, la ecuación \textbf{\eqref{eq_1.56}} se reduce a la ecuación de Stokes-Einstein:

   \begin{equation}
   \mathscr{D}_{AB} = \frac{k_B T}{6 \pi \mu_B R_A}\tag{1.58}\label{eq_1.58}
   \end{equation}

b)-. \( \beta_{AB} = 0 \) (con deslizamiento).  

   En este caso, la ecuación \textbf{\eqref{eq_1.56}} se reduce a:

   \begin{equation}
   \mathscr{D}_{AB} = \frac{k_B T}{4 \pi \mu_B R_A}\tag{1.59}\label{eq_1.59}
   \end{equation}

Si A y B son iguales, en un arreglo cúbico se obtiene:


$2 R_A = \left( \frac{\widetilde{V}_A}{\widetilde{N}_A} \right)^{1/3}$ y luego:

\begin{equation}
\mathscr{D}_{AB} = \frac{k_B T}{2 \pi \mu_A} \left( \frac{\widetilde{V}_A}{\widetilde{N}_A} \right)^{1/3}\tag{1.60}\label{eq_1.60}
\end{equation}

La teoría de Eyring del "estado activado" postula un estado "cuasicristalino" del líquido. Esta teoría deduce el siguiente resultado dentro del contexto de mecánica estadística:

\begin{equation}
\frac{\mathscr{D}_{AB} \mu_B}{(\mathscr{D}_{AB}\mu_B)_{X_A \to 0}} = \left[ 1 + X_A \frac{\widetilde{V}_A}{\widetilde{V}_B} \right] \left( \frac{\partial \ln \gamma_{A}}{d X_A} \right)_{T,P}\tag{1.61}\label{eq_1.61}
\end{equation}
en donde $\widetilde{V}_A$ y $\widetilde{V}_B$ son los volúmenes parciales molares de A y B, $\gamma_A$ es el coeficiente de actividad de A y $\mathscr{D}_{AB}$,$\mu_A$ son la difusividad y viscocidad de la mezcla.

Expresiones empíricas sugeridas para situaciones reales han sido propuestas. Por ejemplo, la ec. de Wilke-Chang:
\begin{equation}
    \mathscr{D}_{AB}=7.4\times10^{-8}\frac{\sqrt{\psi_BM_B}}{\mu \widetilde{V}_A^{0.6}}T
\tag{1.62}\label{eq_1.62}
\end{equation}
donde $\psi_B$ es el "parametro de asociación" del solvente.

$\psi_B=2.6$ para agua y 1.9 para metanol.
$\psi_B=1$ para solventes no asociativos. $\widetilde{V}_A=M_A/\rho$ y $\mu$ esta dada en cp.
\section{Tarea 1}
Bird et all 17.A
\begin{itemize}
\item[1.-] Predicción binaria a baja densidad. 
    Estimar $D_{AB}$  para el sistema metano-etano a 293°K y 1 atm por medio de los siguientes métodos.
    \begin{itemize}
        \item [a)-] Ecuación 1.53
        \item [b)-] Ecuación 1.53 y gráfica Fig 1.51 utilizando las T y P reducidas $T_{r}=\frac{T}{\sqrt{T_{CA}T_{CB}}}$ , $P_{r}=\frac{P}{\sqrt{P_{CA}P_{CB}}}$
        \item [c)-] Ecuación 1.46, 1.48 y 1.49, y los parámetros de Lennard-Joues del Apéndice C.
        \item [d)-] Ecuación 1.46 en los parámetros de Lennard-Jones estimados a partir de las propiedades críticas siguientes:
         $$\frac{\epsilon}{k_{13}}=0.77\sqrt{T_{CA}T_{CB}}  ,  T=\frac{2.44}{2} \left[ \left(\frac{T_{CA}}{P_{CA}}\right)^{1/5}+\left(\frac{T_{CB}}{P_{CB}}\right)^{1/3} \right] $$
    \end{itemize} 
    Respuestas ($cm^2/s$) : a)- 0.152; b)- 0.138; c)- 0.146; d)- 0.138
\item[2.-] Autodifusión de mercurio líquido Bird et all 17 A3.
La difusividad del $Hg^{203}$ en mercurio líquido normal se ha medido con datos de viscosidad y volumen másico. Comparar los datos experimentales con aquellos obtenidos 
por la ec. 160.
    \begin{table}[H]  %Aquí empieza la tabla%
    \centering  %Le digo  que se centre %
    \begin{tabular}{cccc}
    \hline
    \textbf{T(K)} & \textbf{$ D_{AB} $ $cm^2/s$} & \textbf{$\mu$ cP} & \textbf{$\hat{V}$ $cm^3/s$} \\ \hline
            275.7 & $1.52 \times 10^{-5}$ & 1.68 & 0.0736 \\ 
            289.6 & $1.68 \times 10^{-5}$ & 1.56 & 0.0737 \\ 
            364.2 & $2.57 \times 10^{-5}$ & 1.27 & 0.0748 \\ \hline
        \end{tabular}
    \end{table}  

Bird et all 17.A.5    
\item[3.-] Cálculo de la difusividad de una muestra binaria a alta densidad. 
Predecir $pD_{AB}$ para una mezcla equimolar de $N_{2}$ y $C_{2}H_{6}$ a 288.2 K y 40 atm.
    \begin{itemize}
     \item [a)-] Usar el valor de $D_{AB}$ a 1 atm de  0.148 $cm^2/s$ a T=298.2 Ky la gráfica de la Fig.1.51
     \item [b)-] Usar la ecuación 1.55 y la Fig 1.51
    \end{itemize}    
Respuesta a).- $5.8*10^{-6}$ gmol/cms; b).- $5.3*10^{-6}$ gmol/cms
\item[4.-] Prob. 17.A.6 Bird
Difusividad y número de Schmidit para mezclas cloro-aire. 
    \begin{itemize}
    \item [a)-] Predecir $D_{AB}$ para mezclas cloro-aire a 75°F y 1 atm.
    Utilizar los parámetros del Apéndice C.
    \item [b)-] Calcular (a) utilizando la ec. 1.53
    \item [c)-]Utilizar los resultados de (a) para estimar los valores del número de Schmidit para mezclas cloro-aire a 297 K y 1 atm para las siguientes fracciones mol y viscosidades:
    0, $1.83*10^{-4} poises$; 0.25, $1.64*10^{-4} poises$; 0.5, $1.5*10^{-4} poises$; 0.75, $1.39x10^{-4} poises$; 1, $1.31*10^{-4} poises$
    $$ pD_{AB}= \frac{\rho}{RT}D_{AB}; \space  Sc=\frac{\mu}{M_{C}D_{AB}}=\frac{\mu}{(x_{A}M_{A}+x_{B}M_{B})pD_{AB}}$$
\end{itemize}
Respuestas: a).- 0.121 $cm/s$; b).- 0.124 $cm/s$; c).- Sc=1.27, 0.832, 0.602, 0.463, 0.372
\item[5.-] Probl. 17.A.8 Bird.
Corrección para la difusividad a altas densidades. 
El valor medido para $pD_{AB}$ de una mezcla de 80\% mol de $CH_{4}$ Y 20\% mol de $C_{2}H_{6}$ a 313 K y 136 atm es $6x10^-6$ gmol/cms.
Calcular $pD_{AB}$ para esa mezcla a 136 atm y 351 K usando la Fig 1.51.

Respuesta $6.3*10^{-6} gmol/cm s$. Observado $6.33*10^{-6 }gmol/cm s$
\item[6.-]   Probl 17 A.10 Bird 
Cálculo de difusividad de líquidos 
\begin{itemize}
     \item [a)-] Calcular la difusividad de una solución diluida de ácido acético a 12.5°C utilizando la ec. 1.62. La densidad del ácido acético es 0.937 $g/cm^3$ en el punto de ebullición.
     \item [b)-] La difusividad de una solución diluida de metanol a 15°C es $1.28*10^{-5} cm^2/s$. Calcular la difusividad de esa solución a 100°C. 
     Las viscosidades a 15°C y 100°C son 1.14 cp y 0.28cp.
     La viscosidad de la solución diluida es 1.22 cp.
    \end{itemize}
Respuesta (b).- $6.7*10^{-5} cm^2/s$.
\end{itemize}
\newpage


 
\section{Apéndice A}
Ecuaciones de continuidad en varios sistemas de coordenadas

\subsection*{Coordenadas rectangulares:}
\begin{equation} \frac{\partial c_A}{\partial t} + \left( \frac{\partial N_{Az}}{\partial x} + \frac{\partial N_{Ay}}{\partial y} + \frac{\partial N_{Az}}{\partial z} \right) = R_A \tag{A}
\end{equation}

\subsection*{Coordenadas cilíndricas:}
\begin{equation} \frac{\partial c_A}{\partial t} + \left( \frac{1}{r} \frac{\partial}{\partial r} (rN_{Ar}) + \frac{1}{r} \frac{\partial N_{A\theta}}{\partial \theta} + \frac{\partial N_{Az}}{\partial z} \right) = R_A \tag{B} 
\end{equation}

\subsection*{Coordenadas esféricas:}
\begin{equation} \frac{\partial c_A}{\partial t} + \left( \frac{1}{r^2} \frac{\partial}{\partial r} (r^2 N_{Ar}) + \frac{1}{r \sin \theta} \frac{\partial}{\partial \theta} (N_{A\theta} \sin \theta) + \frac{1}{r \sin \theta} \frac{\partial N_{A\phi}}{\partial \phi} \right) = R_A \tag{C}
\end{equation}

\newpage
\section{Apéndice B}
Ecuación de continuidad de A con $\rho$ y $\mathcal{D}_{AB}$ constantes
\subsection*{Coordenadas rectangulares:}
\begin{equation}  \frac{\partial c_A}{\partial t} + \left( v_x \frac{\partial c_A}{\partial x} + v_y \frac{\partial c_A}{\partial y} + v_z \frac{\partial c_A}{\partial z} \right) = \mathcal{D}_{AB} \left( \frac{\partial^2 c_A}{\partial x^2} + \frac{\partial^2 c_A}{\partial y^2} + \frac{\partial^2 c_A}{\partial z^2} \right) + R_A \tag{A}
\end{equation}

\subsection*{Coordenadas cilíndricas:}
\begin{equation} 
    \begin{split} 
    \frac{\partial c_A}{\partial t} + &\left( v_r \frac{\partial c_A}{\partial r} + v_0 \frac{1}{r} \frac{\partial c_A}{\partial \theta} + v_z \frac{\partial c_A}{\partial z} \right) \\
      &= \mathcal{D}_{AB} \left( \frac{1}{r} \frac{\partial}{\partial r} \left( r \frac{\partial c_A}{\partial r} \right) + \frac{1}{r^2} \frac{\partial^2 c_A}{\partial \theta^2} + \frac{\partial^2 c_A}{\partial z^2} \right) + R_A
    \end{split} 
    \tag{B}
\end{equation}

\subsection*{Coordenadas esféricas:}
\begin{equation}
    \begin{split}
        \frac{\partial c_A}{\partial t} &+ \left( v_r \frac{\partial c_A}{\partial r} + v_{\theta} \frac{1}{r} \frac{\partial c_A}{\partial \theta} + v_\phi \frac{1}{r \sin \theta} \frac{\partial c_A}{\partial \phi} \right) \\
        &= \mathcal{D}_{AB} \left( \frac{1}{r^2} \frac{\partial}{\partial r} \left( r^2 \frac{\partial c_A}{\partial r} \right) + \frac{1}{r^2 \sin \theta} \frac{\partial}{\partial \theta} \left( \sin \theta \frac{\partial c_A}{\partial \theta} \right) + \frac{1}{r^2 \sin^2 \theta} \frac{\partial^2 c_A}{\partial \phi^2} \right) + R_A 
    \end{split}
    \tag{C}
\end{equation}
\input{Capitulo1/tablas.tex}
\newpage




  
